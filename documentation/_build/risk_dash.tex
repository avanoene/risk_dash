%% Generated by Sphinx.
\def\sphinxdocclass{report}
\documentclass[letterpaper,10pt,english]{sphinxmanual}
\ifdefined\pdfpxdimen
   \let\sphinxpxdimen\pdfpxdimen\else\newdimen\sphinxpxdimen
\fi \sphinxpxdimen=.75bp\relax

\PassOptionsToPackage{warn}{textcomp}
\usepackage[utf8]{inputenc}
\ifdefined\DeclareUnicodeCharacter
 \ifdefined\DeclareUnicodeCharacterAsOptional
  \DeclareUnicodeCharacter{"00A0}{\nobreakspace}
  \DeclareUnicodeCharacter{"2500}{\sphinxunichar{2500}}
  \DeclareUnicodeCharacter{"2502}{\sphinxunichar{2502}}
  \DeclareUnicodeCharacter{"2514}{\sphinxunichar{2514}}
  \DeclareUnicodeCharacter{"251C}{\sphinxunichar{251C}}
  \DeclareUnicodeCharacter{"2572}{\textbackslash}
 \else
  \DeclareUnicodeCharacter{00A0}{\nobreakspace}
  \DeclareUnicodeCharacter{2500}{\sphinxunichar{2500}}
  \DeclareUnicodeCharacter{2502}{\sphinxunichar{2502}}
  \DeclareUnicodeCharacter{2514}{\sphinxunichar{2514}}
  \DeclareUnicodeCharacter{251C}{\sphinxunichar{251C}}
  \DeclareUnicodeCharacter{2572}{\textbackslash}
 \fi
\fi
\usepackage{cmap}
\usepackage[T1]{fontenc}
\usepackage{amsmath,amssymb,amstext}
\usepackage{babel}
\usepackage{times}
\usepackage[Bjarne]{fncychap}
\usepackage{sphinx}

\usepackage{geometry}

% Include hyperref last.
\usepackage{hyperref}
% Fix anchor placement for figures with captions.
\usepackage{hypcap}% it must be loaded after hyperref.
% Set up styles of URL: it should be placed after hyperref.
\urlstyle{same}
\addto\captionsenglish{\renewcommand{\contentsname}{Contents:}}

\addto\captionsenglish{\renewcommand{\figurename}{Fig.}}
\addto\captionsenglish{\renewcommand{\tablename}{Table}}
\addto\captionsenglish{\renewcommand{\literalblockname}{Listing}}

\addto\captionsenglish{\renewcommand{\literalblockcontinuedname}{continued from previous page}}
\addto\captionsenglish{\renewcommand{\literalblockcontinuesname}{continues on next page}}

\addto\extrasenglish{\def\pageautorefname{page}}

\setcounter{tocdepth}{2}



\title{risk\_dash Documentation}
\date{Jun 03, 2018}
\release{0.0.2}
\author{Alexander van Oene}
\newcommand{\sphinxlogo}{\vbox{}}
\renewcommand{\releasename}{Release}
\makeindex

\begin{document}

\maketitle
\sphinxtableofcontents
\phantomsection\label{\detokenize{index::doc}}

\phantomsection\label{\detokenize{gettingstarted:gettingstarted}}

\chapter{risk\_dash}
\label{\detokenize{gettingstarted:risk-dash}}\label{\detokenize{gettingstarted::doc}}\begin{itemize}
\item {} 
{\hyperref[\detokenize{gettingstarted:overview}]{\emph{Overview}}}

\item {} 
{\hyperref[\detokenize{gettingstarted:getting-started}]{\emph{Getting Started}}}
\begin{itemize}
\item {} 
{\hyperref[\detokenize{gettingstarted:security-data-security-objects-and-creating-security-subclasses}]{\emph{Security data, \_Security objects, and creating Security
Subclasses}}}

\item {} 
{\hyperref[\detokenize{gettingstarted:portfolio-data-and-creating-a-portfolio}]{\emph{Portfolio Data and creating a
Portfolio}}}

\item {} 
{\hyperref[\detokenize{gettingstarted:calculating-risk-metrics-and-using-the-portfolio-class}]{\emph{Calculating Risk Metrics and Using the Portfolio
Class}}}
\begin{itemize}
\item {} 
{\hyperref[\detokenize{gettingstarted:mark-the-portfolio}]{\emph{Mark the Portfolio}}}

\item {} 
{\hyperref[\detokenize{gettingstarted:parametrically-calculating-the-value-at-risk}]{\emph{Parametrically Calculating the Value at
Risk}}}

\item {} 
{\hyperref[\detokenize{gettingstarted:simulating-the-portfolio}]{\emph{Simulating the Portfolio}}}

\end{itemize}

\item {} 
{\hyperref[\detokenize{gettingstarted:summary}]{\emph{Summary}}}

\end{itemize}

\end{itemize}


\section{Overview}
\label{\detokenize{gettingstarted:overview}}
\sphinxhref{https://github.com/avanoene/risk\_dash}{risk\_dash} is a framework to
help simplify the data flow for a portfolio of assets and handle market
risk metrics at the asset and portfolio level. If you clone the source
\sphinxhref{https://github.com/avanoene/risk\_dash}{repository}, included is a
\sphinxhref{https://plot.ly/dash/}{Dash} application to be an example of some of
the uses for the package. To run the Dash app, documentation is
\sphinxhref{dashdocumentation.html}{here}


\section{Installation}
\label{\detokenize{gettingstarted:installation}}
Since the package is in heavy development, to install the package fork
or clone the \sphinxhref{https://github.com/avanoene/risk\_dash}{repository} and
run \sphinxcode{\sphinxupquote{pip install -e risk\_dash/}} from the directory above your local
repository.

To see if installation was successful run
\sphinxcode{\sphinxupquote{python -c 'import risk\_dash; print(*dir(risk\_dash), sep="\textbackslash{}n")'}} in
the command line, currently the output should match the following:

\fvset{hllines={, ,}}%
\begin{sphinxVerbatim}[commandchars=\\\{\}]
\PYGZdl{} python \PYGZhy{}c \PYGZsq{}import risk\PYGZus{}dash; print(*dir(risk\PYGZus{}dash), sep=\PYGZdq{}\PYGZbs{}n\PYGZdq{})\PYGZsq{}
\PYGZus{}\PYGZus{}builtins\PYGZus{}\PYGZus{}
\PYGZus{}\PYGZus{}cached\PYGZus{}\PYGZus{}
\PYGZus{}\PYGZus{}doc\PYGZus{}\PYGZus{}
\PYGZus{}\PYGZus{}file\PYGZus{}\PYGZus{}
\PYGZus{}\PYGZus{}loader\PYGZus{}\PYGZus{}
\PYGZus{}\PYGZus{}name\PYGZus{}\PYGZus{}
\PYGZus{}\PYGZus{}package\PYGZus{}\PYGZus{}
\PYGZus{}\PYGZus{}path\PYGZus{}\PYGZus{}
\PYGZus{}\PYGZus{}spec\PYGZus{}\PYGZus{}
market\PYGZus{}data
name
securities
simgen
\end{sphinxVerbatim}


\section{Getting Started}
\label{\detokenize{gettingstarted:getting-started}}
Now that we have the package installed, let’s go through the object
workflow to construct a simple long/short equity portfolio.

High level, we need to specify:
\begin{enumerate}
\item {} 
Portfolio Data
\begin{itemize}
\item {} 
We need to know what’s in the portfolio
\begin{itemize}
\item {} 
Portfolio weights

\item {} 
Types of Assets/Securities

\end{itemize}

\end{itemize}

\item {} 
Security data
\begin{itemize}
\item {} 
We need to know what is important to financially model the
security
\begin{itemize}
\item {} 
Identification data: Ticker, CUSIP, Exchange

\item {} 
Security specific data: expiry, valuation functions

\item {} 
Market data: Closing prices, YTM

\end{itemize}

\end{itemize}

\item {} 
Portfolio/security constructors to handle the above data

\end{enumerate}

To visualize these constructors, the below chart shows how the data will
sit:

\sphinxincludegraphics{{cbf3d94243da296ec3347cf846f31097392405d0}.png}

To do so, we’ll need subclasses for the {\hyperref[\detokenize{gettingstarted:security}]{\emph{\_Security}}} and
\sphinxhref{buildingclasses.html}{\_MarketData} classes to model specific types
of securities. Currently supported is the Equity subclass. Once we have
the portfolio constructed, we will specify and calculate parameters to
simulate or look at historic distributions. We’ll then create a subclass
of {\hyperref[\detokenize{gettingstarted:simulation}]{\emph{\_Simulation}}} and {\hyperref[\detokenize{gettingstarted:randomgen}]{\emph{\_RandomGen}}}


\subsection{Security data, \_Security objects, and creating Security Subclasses}
\label{\detokenize{gettingstarted:security-data-security-objects-and-creating-security-subclasses}}
The core of the package is in the \_Security and Portfolio objects.
Portfolio objects are naturally a collection of Securities, however we
want to specify the type of securities that are in the portfolio. Since
we’re focusing on a long/short equity portfolio we want to create an
Equity subclass.

Subclasses of \_Security classes must have the following methods:
\begin{itemize}
\item {} 
valuation(current\_price)

\item {} 
mark\_to\_market(current\_price)

\item {} 
get\_marketdata()

\end{itemize}

In addition, we want to pass them the associated \_MarketData object to
represent the security’s historic pricing data. To build the Equity
subclass, we first want to inherit any methods from the \_Security
class:

\fvset{hllines={, ,}}%
\begin{sphinxVerbatim}[commandchars=\\\{\}]
\PYG{k}{class} \PYG{n+nc}{Equity}\PYG{p}{(}\PYG{n}{\PYGZus{}Security}\PYG{p}{)}\PYG{p}{:}

    \PYG{k}{def} \PYG{n+nf}{\PYGZus{}\PYGZus{}init\PYGZus{}\PYGZus{}}\PYG{p}{(}
            \PYG{n+nb+bp}{self}\PYG{p}{,}
            \PYG{n}{ticker}\PYG{p}{,}
            \PYG{n}{market\PYGZus{}data} \PYG{p}{:} \PYG{n}{md}\PYG{o}{.}\PYG{n}{QuandlStockData}\PYG{p}{,}
            \PYG{n}{ordered\PYGZus{}price}\PYG{p}{,}
            \PYG{n}{quantity}\PYG{p}{,}
            \PYG{n}{date\PYGZus{}ordered}
        \PYG{p}{)}\PYG{p}{:}
        \PYG{n+nb+bp}{self}\PYG{o}{.}\PYG{n}{name} \PYG{o}{=} \PYG{n}{ticker}
        \PYG{n+nb+bp}{self}\PYG{o}{.}\PYG{n}{market\PYGZus{}data} \PYG{o}{=} \PYG{n}{market\PYGZus{}data}
        \PYG{n+nb+bp}{self}\PYG{o}{.}\PYG{n}{ordered\PYGZus{}price} \PYG{o}{=} \PYG{n}{ordered\PYGZus{}price}
        \PYG{n+nb+bp}{self}\PYG{o}{.}\PYG{n}{quantity} \PYG{o}{=} \PYG{n}{quantity}
        \PYG{n+nb+bp}{self}\PYG{o}{.}\PYG{n}{initial\PYGZus{}value} \PYG{o}{=} \PYG{n}{ordered\PYGZus{}price} \PYG{o}{*} \PYG{n}{quantity}
        \PYG{n+nb+bp}{self}\PYG{o}{.}\PYG{n}{date\PYGZus{}ordered} \PYG{o}{=} \PYG{n}{date\PYGZus{}ordered}
        \PYG{n+nb+bp}{self}\PYG{o}{.}\PYG{n}{type} \PYG{o}{=} \PYG{l+s+s1}{\PYGZsq{}}\PYG{l+s+s1}{Equity}\PYG{l+s+s1}{\PYGZsq{}}
\end{sphinxVerbatim}

To break down the inputs, we want to keep in mind that the goal of this
subclass of the \_Security object is to provide an interface to model
the Equity data.
\begin{itemize}
\item {} 
ticker is going to be the ticker code for the equity, such as ‘AAPL’

\item {} 
market\_data is going to be a subclass of the \_MarketData object

\item {} 
ordered\_price is going to be the price which the trade occurred

\item {} 
quantity for Equity will be the number of shares

\item {} 
date\_ordered should be the date the order was placed

\end{itemize}

\begin{sphinxadmonition}{note}{Note:}
Currently the implemented \_MarketData subclass is
QuandlStockData, which is a wrapper for \sphinxhref{https://www.quandl.com/databases/WIKIP}{this Quandl dataset
api}. This data is no
longer being updated, for current market prices you must create a
\_MarketData subclass for your particular market data. Information
to construct the subclass is on \sphinxhref{buildingclasses.html}{Building Custom
Classes}.
\end{sphinxadmonition}

Required Inputs at the \_Security level are intentionally limited, for
example if we wanted to create a class for Fixed Income securities, we
would want more information than this Equity subclass. An example Bond
class might look like this:

\fvset{hllines={, ,}}%
\begin{sphinxVerbatim}[commandchars=\\\{\}]
\PYG{k}{class} \PYG{n+nc}{Bond}\PYG{p}{(}\PYG{n}{\PYGZus{}Security}\PYG{p}{)}\PYG{p}{:}
    \PYG{k}{def} \PYG{n+nf}{\PYGZus{}\PYGZus{}init\PYGZus{}\PYGZus{}}\PYG{p}{(}
            \PYG{n+nb+bp}{self}\PYG{p}{,}
            \PYG{n}{CUSIP}\PYG{p}{,}
            \PYG{n}{market\PYGZus{}data}\PYG{p}{,}
            \PYG{n}{expiry}\PYG{p}{,}
            \PYG{n}{coupon}\PYG{p}{,}
            \PYG{n}{frequency}\PYG{p}{,}
            \PYG{n}{settlement\PYGZus{}date}\PYG{p}{,}
            \PYG{n}{face\PYGZus{}value}
        \PYG{p}{)}\PYG{p}{:}
        \PYG{n+nb+bp}{self}\PYG{o}{.}\PYG{n}{name} \PYG{o}{=} \PYG{n}{CUSIP}
        \PYG{n+nb+bp}{self}\PYG{o}{.}\PYG{n}{market\PYGZus{}data} \PYG{o}{=} \PYG{n}{market\PYGZus{}data}
        \PYG{n+nb+bp}{self}\PYG{o}{.}\PYG{n}{expiry} \PYG{o}{=} \PYG{n}{expiry}
        \PYG{n+nb+bp}{self}\PYG{o}{.}\PYG{n}{coupon} \PYG{o}{=} \PYG{n}{coupon}
        \PYG{n+nb+bp}{self}\PYG{o}{.}\PYG{n}{frequency} \PYG{o}{=} \PYG{n}{frequency}
        \PYG{n+nb+bp}{self}\PYG{o}{.}\PYG{n}{settlement\PYGZus{}date} \PYG{o}{=} \PYG{n}{settlement\PYGZus{}date}
        \PYG{n+nb+bp}{self}\PYG{o}{.}\PYG{n}{face\PYGZus{}value} \PYG{o}{=} \PYG{n}{face\PYGZus{}value}
        \PYG{n+nb+bp}{self}\PYG{o}{.}\PYG{n}{type} \PYG{o}{=} \PYG{l+s+s1}{\PYGZsq{}}\PYG{l+s+s1}{Bond}\PYG{l+s+s1}{\PYGZsq{}}
\end{sphinxVerbatim}

Similarly to the Equity subclass, we want identification information,
market data, and arguments that will either help in calculating
valuation, current returns, or risk measures.

Returning to the Equity subclass, we now need to write the valuation and
mark to market methods:

\fvset{hllines={, ,}}%
\begin{sphinxVerbatim}[commandchars=\\\{\}]
\PYG{k}{class} \PYG{n+nc}{Equity}\PYG{p}{(}\PYG{n}{\PYGZus{}Security}\PYG{p}{)}\PYG{p}{:}
  \PYG{c+c1}{\PYGZsh{} ...}
  \PYG{k}{def} \PYG{n+nf}{valuation}\PYG{p}{(}\PYG{n+nb+bp}{self}\PYG{p}{,} \PYG{n}{price}\PYG{p}{)}\PYG{p}{:}
      \PYG{n}{value} \PYG{o}{=} \PYG{p}{(}\PYG{n}{price} \PYG{o}{\PYGZhy{}} \PYG{n+nb+bp}{self}\PYG{o}{.}\PYG{n}{ordered\PYGZus{}price}\PYG{p}{)} \PYG{o}{*} \PYG{n+nb+bp}{self}\PYG{o}{.}\PYG{n}{quantity}
      \PYG{k}{return}\PYG{p}{(}\PYG{n}{value}\PYG{p}{)}

  \PYG{k}{def} \PYG{n+nf}{mark\PYGZus{}to\PYGZus{}market}\PYG{p}{(}\PYG{n+nb+bp}{self}\PYG{p}{,} \PYG{n}{current\PYGZus{}price}\PYG{p}{)}\PYG{p}{:}
      \PYG{n+nb+bp}{self}\PYG{o}{.}\PYG{n}{market\PYGZus{}value} \PYG{o}{=} \PYG{n+nb+bp}{self}\PYG{o}{.}\PYG{n}{quantity} \PYG{o}{*} \PYG{n}{current\PYGZus{}price}
      \PYG{n+nb+bp}{self}\PYG{o}{.}\PYG{n}{marked\PYGZus{}change} \PYG{o}{=} \PYG{n+nb+bp}{self}\PYG{o}{.}\PYG{n}{valuation}\PYG{p}{(}\PYG{n}{current\PYGZus{}price}\PYG{p}{)}
      \PYG{k}{return}\PYG{p}{(}\PYG{n+nb+bp}{self}\PYG{o}{.}\PYG{n}{marked\PYGZus{}change}\PYG{p}{)}
\end{sphinxVerbatim}

For linear instruments such as equities, valuation of a position is just
the price observed minus the price ordered at the size of the position.
\sphinxcode{\sphinxupquote{valuation}} is then used to pass a hypothetical price into the
valuation function, in this case (Price - Ordered) * Quantity, where as
\sphinxcode{\sphinxupquote{mark\_to\_market}} is used to pass the current EOD price and mark the
value of the position. This is an important distinction, if we had a
nonlinear instrument such as a call option on a company’s equity price,
the valuation function would then be:
\begin{equation*}
\begin{split}Value = min\{0, S_{T} - K\}\end{split}
\end{equation*}
Where S_{T} is the spot price for the equity at expiry and
K is the agreed strike price. Valuation also is dependent on
time for option data, however if you were to use a binomial tree to
evaluate the option, you would want to use this same value function and
discount the value a each node back to time=0.

Our mark to market then would need to make the distinction between this
valuation and the current market price for the call option. The mark
would then keep track of what the current market value for the option to
keep track of actualized returns.

The final piece to creating the Equity subclass is then to add a
\sphinxcode{\sphinxupquote{get\_marketdata()}}method. Since we just want a copy of the reference
of the \sphinxcode{\sphinxupquote{market\_data}}, we can just inherit the \sphinxcode{\sphinxupquote{get\_marketdata()}}
from the \_Security class.

The Equity subclass is already implemented in the package, we can create
an instance from \sphinxcode{\sphinxupquote{risk\_dash.securities}}. Let’s make an instance that
represents an order of 50 shares of AAPL, Apple Inc, at close on March
9th, 2018:

\fvset{hllines={, ,}}%
\begin{sphinxVerbatim}[commandchars=\\\{\}]
\PYG{g+gp}{\PYGZgt{}\PYGZgt{}\PYGZgt{} }\PYG{k+kn}{from} \PYG{n+nn}{risk\PYGZus{}dash}\PYG{n+nn}{.}\PYG{n+nn}{market\PYGZus{}data} \PYG{k}{import} \PYG{n}{QuandlStockData}
\PYG{g+gp}{\PYGZgt{}\PYGZgt{}\PYGZgt{} }\PYG{k+kn}{from} \PYG{n+nn}{risk\PYGZus{}dash}\PYG{n+nn}{.}\PYG{n+nn}{securities} \PYG{k}{import} \PYG{n}{Equity}
\PYG{g+gp}{\PYGZgt{}\PYGZgt{}\PYGZgt{} }\PYG{k+kn}{from} \PYG{n+nn}{datetime} \PYG{k}{import} \PYG{n}{datetime}
\PYG{g+gp}{\PYGZgt{}\PYGZgt{}\PYGZgt{} }\PYG{n}{apikey} \PYG{o}{=} \PYG{l+s+s1}{\PYGZsq{}}\PYG{l+s+s1}{valid\PYGZhy{}quandl\PYGZhy{}apikey}\PYG{l+s+s1}{\PYGZsq{}}
\PYG{g+gp}{\PYGZgt{}\PYGZgt{}\PYGZgt{} }\PYG{n}{aapl\PYGZus{}market\PYGZus{}data} \PYG{o}{=} \PYG{n}{QuandlStockData}\PYG{p}{(}
\PYG{g+go}{  apikey = apikey,}
\PYG{g+go}{  ticker = \PYGZsq{}AAPL\PYGZsq{}}
\PYG{g+go}{)}
\PYG{g+gp}{\PYGZgt{}\PYGZgt{}\PYGZgt{} }\PYG{n}{aapl\PYGZus{}stock} \PYG{o}{=} \PYG{n}{Equity}\PYG{p}{(}
\PYG{g+go}{  ticker = \PYGZsq{}AAPL\PYGZsq{},}
\PYG{g+go}{  market\PYGZus{}data = aapl\PYGZus{}market\PYGZus{}data,}
\PYG{g+go}{  ordered\PYGZus{}price = 179.98,}
\PYG{g+go}{  quantity = 50,}
\PYG{g+go}{  date\PYGZus{}ordered = datetime(2018,3,9)}
\PYG{g+go}{)}
\PYG{g+gp}{\PYGZgt{}\PYGZgt{}\PYGZgt{} }\PYG{n}{aapl\PYGZus{}stock}\PYG{o}{.}\PYG{n}{valuation}\PYG{p}{(}\PYG{l+m+mf}{180.98}\PYG{p}{)} \PYG{c+c1}{\PYGZsh{} \PYGZdl{}1 increase in value}
\PYG{g+go}{50.0}
\PYG{g+gp}{\PYGZgt{}\PYGZgt{}\PYGZgt{} }\PYG{n}{aapl\PYGZus{}stock}\PYG{o}{.}\PYG{n}{mark\PYGZus{}to\PYGZus{}market}\PYG{p}{(}\PYG{l+m+mf}{180.98}\PYG{p}{)} \PYG{c+c1}{\PYGZsh{} Same \PYGZdl{}1 increase}
\PYG{g+go}{50.0}
\PYG{g+gp}{\PYGZgt{}\PYGZgt{}\PYGZgt{} }\PYG{n}{aapl\PYGZus{}stock}\PYG{o}{.}\PYG{n}{market\PYGZus{}value}
\PYG{g+go}{9049.0}
\PYG{g+gp}{\PYGZgt{}\PYGZgt{}\PYGZgt{} }\PYG{n}{aapl\PYGZus{}stock}\PYG{o}{.}\PYG{n}{marked\PYGZus{}change}
\PYG{g+go}{50.0}
\PYG{g+gp}{\PYGZgt{}\PYGZgt{}\PYGZgt{} }\PYG{n+nb}{vars}\PYG{p}{(}\PYG{n}{aapl\PYGZus{}stock}\PYG{p}{)}
\PYG{g+go}{\PYGZob{}\PYGZsq{}name\PYGZsq{}: \PYGZsq{}AAPL\PYGZsq{},}
\PYG{g+go}{ \PYGZsq{}market\PYGZus{}data\PYGZsq{}: \PYGZlt{}risk\PYGZus{}dash.market\PYGZus{}data.QuandlStockData at 0x1147c2668\PYGZgt{},}
\PYG{g+go}{ \PYGZsq{}ordered\PYGZus{}price\PYGZsq{}: 179.98,}
\PYG{g+go}{ \PYGZsq{}quantity\PYGZsq{}: 50,}
\PYG{g+go}{ \PYGZsq{}initial\PYGZus{}value\PYGZsq{}: 8999.0,}
\PYG{g+go}{ \PYGZsq{}date\PYGZus{}ordered\PYGZsq{}: datetime.datetime(2018, 3, 9, 0, 0),}
\PYG{g+go}{ \PYGZsq{}type\PYGZsq{}: \PYGZsq{}Equity\PYGZsq{},}
\PYG{g+go}{ \PYGZsq{}market\PYGZus{}value\PYGZsq{}: 9049.0,}
\PYG{g+go}{ \PYGZsq{}marked\PYGZus{}change\PYGZsq{}: 50.0\PYGZcb{}}
\end{sphinxVerbatim}

As we can see \sphinxcode{\sphinxupquote{aapl\_stock}} now is a container that we can use to
access it’s attributes at the Portfolio level.

\begin{sphinxadmonition}{note}{Note:}
Another important observation is that the Equity subclass will
only keep a reference to the underlying QuandlStockData, which will
minimize duplication of data. However, at scale, you’d want minimize
price calls to your data source, you could then do one call at the
Portfolio level then pass a reference to that market\_data at the
individual level. Then your Equity or other \_Security subclasses
can share the same \_MarketData, you would then just write methods
to interact with that data.
\end{sphinxadmonition}

Now that we have a feeling for the \_Security class, we now want to
build a Portfolio that contains the \_Security instances.


\subsection{Portfolio Data and creating a Portfolio}
\label{\detokenize{gettingstarted:portfolio-data-and-creating-a-portfolio}}
To iterate on what we said before, an equity position in your portfolio
is represented by the quantity you ordered, the price ordered at, and
when you ordered or settled the position. In this example, we’ll use the
following theoretical portfolio found in \sphinxcode{\sphinxupquote{portfolio\_example.csv}}:


\begin{savenotes}\sphinxattablestart
\centering
\begin{tabulary}{\linewidth}[t]{|T|T|T|T|T|}
\hline
\sphinxstyletheadfamily 
Type
&\sphinxstyletheadfamily 
Ticker
&\sphinxstyletheadfamily 
Ordered Price
&\sphinxstyletheadfamily 
Ordered Date
&\sphinxstyletheadfamily 
Quantity
\\
\hline
Equity
&
AAPL
&
179.98
&
3/9/18
&
50
\\
\hline
Equity
&
AMD
&
11.7
&
3/9/18
&
100
\\
\hline
Equity
&
INTC
&
52.19
&
3/9/18
&
-50
\\
\hline
Equity
&
GOOG
&
1160.04
&
3/9/18
&
5
\\
\hline
\end{tabulary}
\par
\sphinxattableend\end{savenotes}

With this example, the portfolio is static, or just one snap shot of the
weights at a given time. In practice, it might be useful to have
multiple snapshots of your portfolio, one’s portfolio would be changing
as positions enter and leave thus having a time dimensionality. The
Portfolio class could be easily adapted to handle that information to
accurately plot historic performance by remarking through time. This
seems more of an accounting exercise, risk metrics looking forward would
probably still only want to account for the current positions in the
portfolio. Due to this insight, the current Portfolio class only looks
at one snap shot in time.

With a portfolio so small, it is very easily stored in a csv and each
security can store the reference to the underlying market data
independently. As such, there is an included portfolio constructor
method in the portfolio class from csv, \sphinxcode{\sphinxupquote{construct\_portfolio\_csv}}:

\fvset{hllines={, ,}}%
\begin{sphinxVerbatim}[commandchars=\\\{\}]
\PYG{g+gp}{\PYGZgt{}\PYGZgt{}\PYGZgt{} }\PYG{k+kn}{from} \PYG{n+nn}{risk\PYGZus{}dash}\PYG{n+nn}{.}\PYG{n+nn}{securities} \PYG{k}{import} \PYG{n}{Portfolio}
\PYG{g+gp}{\PYGZgt{}\PYGZgt{}\PYGZgt{} }\PYG{n}{current\PYGZus{}portfolio} \PYG{o}{=} \PYG{n}{Portfolio}\PYG{p}{(}\PYG{p}{)}
\PYG{g+gp}{\PYGZgt{}\PYGZgt{}\PYGZgt{} }\PYG{n}{port\PYGZus{}dict} \PYG{o}{=} \PYG{n}{current\PYGZus{}portfolio}\PYG{o}{.}\PYG{n}{construct\PYGZus{}portfolio\PYGZus{}csv}\PYG{p}{(}
\PYG{g+go}{  data\PYGZus{}input=\PYGZsq{}portfolio\PYGZus{}example.csv\PYGZsq{},}
\PYG{g+go}{  apikey=apikey}
\PYG{g+go}{)}
\PYG{g+gp}{\PYGZgt{}\PYGZgt{}\PYGZgt{} }\PYG{n+nb}{vars}\PYG{p}{(}\PYG{n}{current\PYGZus{}portfolio}\PYG{p}{)}
\PYG{g+go}{\PYGZob{}\PYGZsq{}port\PYGZsq{}: \PYGZob{}\PYGZsq{}AAPL Equity\PYGZsq{}: \PYGZlt{}risk\PYGZus{}dash.securities.Equity at 0x11648b5c0\PYGZgt{},}
\PYG{g+go}{  \PYGZsq{}AMD Equity\PYGZsq{}: \PYGZlt{}risk\PYGZus{}dash.securities.Equity at 0x116442c50\PYGZgt{},}
\PYG{g+go}{  \PYGZsq{}INTC Equity\PYGZsq{}: \PYGZlt{}risk\PYGZus{}dash.securities.Equity at 0x1177b75c0\PYGZgt{},}
\PYG{g+go}{  \PYGZsq{}GOOG Equity\PYGZsq{}: \PYGZlt{}risk\PYGZus{}dash.securities.Equity at 0x1177bc390\PYGZgt{}\PYGZcb{}\PYGZcb{}}
\PYG{g+gp}{\PYGZgt{}\PYGZgt{}\PYGZgt{} }\PYG{n+nb}{vars}\PYG{p}{(}\PYG{n}{current\PYGZus{}portfolio}\PYG{o}{.}\PYG{n}{port}\PYG{p}{[}\PYG{l+s+s1}{\PYGZsq{}}\PYG{l+s+s1}{AMD Equity}\PYG{l+s+s1}{\PYGZsq{}}\PYG{p}{]}\PYG{p}{)}
\PYG{g+go}{\PYGZob{}\PYGZsq{}name\PYGZsq{}: \PYGZsq{}AMD\PYGZsq{},}
\PYG{g+go}{ \PYGZsq{}market\PYGZus{}data\PYGZsq{}: \PYGZlt{}risk\PYGZus{}dash.market\PYGZus{}data.QuandlStockData at 0x11648b2e8\PYGZgt{},}
\PYG{g+go}{ \PYGZsq{}ordered\PYGZus{}price\PYGZsq{}: 11.699999999999999,}
\PYG{g+go}{ \PYGZsq{}quantity\PYGZsq{}: 100,}
\PYG{g+go}{ \PYGZsq{}initial\PYGZus{}value\PYGZsq{}: 1170.0,}
\PYG{g+go}{ \PYGZsq{}date\PYGZus{}ordered\PYGZsq{}: \PYGZsq{}3/9/18\PYGZsq{},}
\PYG{g+go}{ \PYGZsq{}type\PYGZsq{}: \PYGZsq{}Equity\PYGZsq{}\PYGZcb{}}
\end{sphinxVerbatim}

At this moment, the \sphinxcode{\sphinxupquote{current\_portfolio}} instance is only a wrapper for
it’s port attribute, a dictionary containing the securities in the
Portfolio object. Soon we’ll use this object to mark the portfolio,
create a simulation to estimate value at risk, look at the covariance
variance matrix to calculate a parameterized volatility measure, and
much more.

The \sphinxcode{\sphinxupquote{Portfolio}} class handles interactions with the portfolio data and
the associated securities in the portfolio. If you have a list of
securities you can also just pass the list into the Portfolio instance.
The following code creates a portfolio of just the AAPL equity that we
created earlier:

\fvset{hllines={, ,}}%
\begin{sphinxVerbatim}[commandchars=\\\{\}]
\PYG{g+gp}{\PYGZgt{}\PYGZgt{}\PYGZgt{} }\PYG{n}{aapl\PYGZus{}portfolio} \PYG{o}{=} \PYG{n}{sec}\PYG{o}{.}\PYG{n}{Portfolio}\PYG{p}{(}\PYG{p}{[}\PYG{n}{aapl\PYGZus{}stock}\PYG{p}{]}\PYG{p}{)}
\PYG{g+gp}{\PYGZgt{}\PYGZgt{}\PYGZgt{} }\PYG{n+nb}{vars}\PYG{p}{(}\PYG{n}{aapl\PYGZus{}portfolio}\PYG{p}{)}
\PYG{g+go}{\PYGZob{}\PYGZsq{}port\PYGZsq{}: \PYGZob{}\PYGZsq{}AAPL Equity\PYGZsq{}: \PYGZlt{}risk\PYGZus{}dash.securities.Equity at 0x1164b2e80\PYGZgt{}\PYGZcb{}\PYGZcb{}}
\end{sphinxVerbatim}

If we want to add a security to this portfolio, we can call the
\sphinxcode{\sphinxupquote{add\_security}} method, to remove a security we call the
\sphinxcode{\sphinxupquote{remove\_security}} method:

\fvset{hllines={, ,}}%
\begin{sphinxVerbatim}[commandchars=\\\{\}]
\PYG{g+gp}{\PYGZgt{}\PYGZgt{}\PYGZgt{} }\PYG{n}{amd\PYGZus{}market\PYGZus{}data} \PYG{o}{=} \PYG{n}{sec}\PYG{o}{.}\PYG{n}{QuandlStockData}\PYG{p}{(}
\PYG{g+go}{  ticker=\PYGZsq{}AMD\PYGZsq{},}
\PYG{g+go}{  apikey=apikey}
\PYG{g+go}{)}
\PYG{g+gp}{\PYGZgt{}\PYGZgt{}\PYGZgt{} }\PYG{n}{amd\PYGZus{}stock} \PYG{o}{=} \PYG{n}{sec}\PYG{o}{.}\PYG{n}{Equity}\PYG{p}{(}
\PYG{g+go}{  ticker = \PYGZsq{}AAPL\PYGZsq{},}
\PYG{g+go}{  market\PYGZus{}data = amd\PYGZus{}market\PYGZus{}data,}
\PYG{g+go}{  ordered\PYGZus{}price = 11.70,}
\PYG{g+go}{  quantity = 100,}
\PYG{g+go}{  date\PYGZus{}ordered = datetime(2018,3,9)}
\PYG{g+go}{)}
\PYG{g+gp}{\PYGZgt{}\PYGZgt{}\PYGZgt{} }\PYG{n}{aapl\PYGZus{}portfolio}\PYG{o}{.}\PYG{n}{add\PYGZus{}security}\PYG{p}{(}\PYG{n}{amd\PYGZus{}stock}\PYG{p}{)}
\PYG{g+gp}{\PYGZgt{}\PYGZgt{}\PYGZgt{} }\PYG{n}{aapl\PYGZus{}portfolio}\PYG{o}{.}\PYG{n}{port}
\PYG{g+go}{\PYGZob{}\PYGZsq{}AAPL Equity\PYGZsq{}: \PYGZlt{}risk\PYGZus{}dash.securities.Equity at 0x1164b2e80\PYGZgt{},}
\PYG{g+go}{ \PYGZsq{}AMD Equity\PYGZsq{}: \PYGZlt{}risk\PYGZus{}dash.securities.Equity at 0x11791cc88\PYGZgt{}\PYGZcb{}}
\PYG{g+gp}{\PYGZgt{}\PYGZgt{}\PYGZgt{} }\PYG{n}{aapl\PYGZus{}portfolio}\PYG{o}{.}\PYG{n}{remove\PYGZus{}security}\PYG{p}{(}\PYG{n}{amd\PYGZus{}stock}\PYG{p}{)}
\PYG{g+gp}{\PYGZgt{}\PYGZgt{}\PYGZgt{} }\PYG{n}{aapl\PYGZus{}portfolio}\PYG{o}{.}\PYG{n}{port}
\PYG{g+go}{\PYGZob{}\PYGZsq{}AAPL Equity\PYGZsq{}: \PYGZlt{}risk\PYGZus{}dash.securities.Equity at 0x1164b2e80\PYGZgt{}\PYGZcb{}}
\PYG{g+gp}{\PYGZgt{}\PYGZgt{}\PYGZgt{} }\PYG{n}{aapl\PYGZus{}portfolio}\PYG{o}{.}\PYG{n}{remove\PYGZus{}security}\PYG{p}{(}\PYG{n}{aapl\PYGZus{}stock}\PYG{p}{)}
\PYG{g+gp}{\PYGZgt{}\PYGZgt{}\PYGZgt{} }\PYG{n}{aapl\PYGZus{}portfolio}\PYG{o}{.}\PYG{n}{port}
\PYG{g+go}{\PYGZob{}\PYGZcb{}}
\end{sphinxVerbatim}


\subsection{Calculating Risk Metrics and Using the Portfolio class}
\label{\detokenize{gettingstarted:calculating-risk-metrics-and-using-the-portfolio-class}}
Now that we have our \sphinxcode{\sphinxupquote{Portfolio}} constructed with the securities we
have on the book let’s use the class to calculate some market risk
metrics.


\subsubsection{Mark the Portfolio}
\label{\detokenize{gettingstarted:mark-the-portfolio}}
Let’s first mark the current portfolio. Since we want to know the
current value of the portfolio, the mark method will calculate the value
of the portfolio at the current price for each security. The current
price is going to be the last known mark, the price at the closest date
to today.

\begin{sphinxadmonition}{note}{Note:}
Since the QuandlStockData source hasn’t been updated since
3/27/2018, we would expect the last shared date to be 3/27/2018.
However, you should use the last shared date as a flag to see if an
asset’s \_MarketData isn’t updating. With certain assets, such as
Bonds or illiquid securities, marking daily might not make as much
sense, so common shared date doesn’t mean as much.
\end{sphinxadmonition}

\fvset{hllines={, ,}}%
\begin{sphinxVerbatim}[commandchars=\\\{\}]
\PYG{g+gp}{\PYGZgt{}\PYGZgt{}\PYGZgt{} }\PYG{n}{current\PYGZus{}portfolio}\PYG{o}{.}\PYG{n}{mark}\PYG{p}{(}\PYG{p}{)}
\PYG{g+gp}{\PYGZgt{}\PYGZgt{}\PYGZgt{} }\PYG{n+nb}{vars}\PYG{p}{(}\PYG{n}{current\PYGZus{}portfolio}\PYG{p}{)}
\PYG{g+go}{\PYGZob{}\PYGZsq{}port\PYGZsq{}: \PYGZob{}\PYGZsq{}AAPL Equity\PYGZsq{}: \PYGZlt{}risk\PYGZus{}dash.securities.Equity at 0x10f8b2940\PYGZgt{},}
\PYG{g+go}{  \PYGZsq{}AMD Equity\PYGZsq{}: \PYGZlt{}risk\PYGZus{}dash.securities.Equity at 0x1a1f6b0908\PYGZgt{},}
\PYG{g+go}{  \PYGZsq{}INTC Equity\PYGZsq{}: \PYGZlt{}risk\PYGZus{}dash.securities.Equity at 0x110538d30\PYGZgt{},}
\PYG{g+go}{  \PYGZsq{}GOOG Equity\PYGZsq{}: \PYGZlt{}risk\PYGZus{}dash.securities.Equity at 0x110548e10\PYGZgt{}\PYGZcb{},}
\PYG{g+go}{ \PYGZsq{}market\PYGZus{}change\PYGZsq{}: \PYGZhy{}1476.6999999999989,}
\PYG{g+go}{ \PYGZsq{}marked\PYGZus{}portfolio\PYGZsq{}: \PYGZob{}\PYGZsq{}AAPL Equity\PYGZsq{}: (8999.0, 8417.0),}
\PYG{g+go}{  \PYGZsq{}AMD Equity\PYGZsq{}: (1170.0, 1000.0),}
\PYG{g+go}{  \PYGZsq{}INTC Equity\PYGZsq{}: (\PYGZhy{}2609.5, \PYGZhy{}2559.5),}
\PYG{g+go}{  \PYGZsq{}GOOG Equity\PYGZsq{}: (5800.1999999999998, 5025.5)\PYGZcb{},}
\PYG{g+go}{ \PYGZsq{}date\PYGZus{}marked\PYGZsq{}: Timestamp(\PYGZsq{}2018\PYGZhy{}03\PYGZhy{}27 00:00:00\PYGZsq{}),}
\PYG{g+go}{ \PYGZsq{}initial\PYGZus{}value\PYGZsq{}: 13359.700000000001\PYGZcb{}}
\end{sphinxVerbatim}

The \sphinxcode{\sphinxupquote{mark}} method now creates the \sphinxcode{\sphinxupquote{marked\_portfolio}} dictionary that
stores a tuple, (initial\_value, market\_value), for every security in
the portfolio. We also now can calculate a quick holding period return,
\sphinxcode{\sphinxupquote{holdingreturn = (current\_portfolio.initial\_value + current\_portfolio.market\_change)/current\_portfolio.initial\_value}}

\fvset{hllines={, ,}}%
\begin{sphinxVerbatim}[commandchars=\\\{\}]
\PYG{g+gp}{\PYGZgt{}\PYGZgt{}\PYGZgt{} }\PYG{n}{holdingreturn} \PYG{o}{=} \PYG{p}{(}\PYG{n}{current\PYGZus{}portfolio}\PYG{o}{.}\PYG{n}{market\PYGZus{}change}\PYG{p}{)}\PYG{o}{/}\PYG{n}{current\PYGZus{}portfolio}\PYG{o}{.}\PYG{n}{initial\PYGZus{}value}
\PYG{g+gp}{\PYGZgt{}\PYGZgt{}\PYGZgt{} }\PYG{n+nb}{print}\PYG{p}{(}\PYG{n}{holdingreturn}\PYG{p}{)}
\PYG{g+go}{\PYGZhy{}0.11053391917483169}
\end{sphinxVerbatim}

This hypothetical portfolio apparently hasn’t performed over the month
since inception, it’s lost 11\%, but let’s look at historic returns
before we give up on the portfolio. We can call \sphinxcode{\sphinxupquote{portfolio.quick\_plot}}
to look at a \sphinxcode{\sphinxupquote{matplotlib}} generated cumulative return series of the
portfolio. If you wanted more control over plotting, you could use the
returned
\sphinxcode{\sphinxupquote{pandas DataFrame. In fact, the current implementation is just using the}}pandas
DataFrame\sphinxcode{\sphinxupquote{method}}plot(){}`:

\fvset{hllines={, ,}}%
\begin{sphinxVerbatim}[commandchars=\\\{\}]
\PYG{g+gp}{\PYGZgt{}\PYGZgt{}\PYGZgt{} }\PYG{n}{marketdata} \PYG{o}{=} \PYG{n}{current\PYGZus{}portfolio}\PYG{o}{.}\PYG{n}{quick\PYGZus{}plot}\PYG{p}{(}\PYG{p}{)}
\end{sphinxVerbatim}

\begin{figure}[htbp]
\centering
\capstart

\noindent\sphinxincludegraphics{{quick_plot_image}.png}
\caption{quick\_plot() Output}\label{\detokenize{gettingstarted:id1}}\end{figure}


\subsubsection{Parametrically Calculating the Value at Risk}
\label{\detokenize{gettingstarted:parametrically-calculating-the-value-at-risk}}
As we can see, this portfolio is pretty volatile, but has almost doubled
over the last four years. Let’s calculate what the portfolio daily
volatility over the period based off the percent change by calling
\sphinxcode{\sphinxupquote{get\_port\_volatility}} using \sphinxcode{\sphinxupquote{percentchange}} from the
\sphinxcode{\sphinxupquote{market\_data}}:

\fvset{hllines={, ,}}%
\begin{sphinxVerbatim}[commandchars=\\\{\}]
\PYG{g+gp}{\PYGZgt{}\PYGZgt{}\PYGZgt{} }\PYG{n}{variance}\PYG{p}{,} \PYG{n}{value\PYGZus{}at\PYGZus{}risk} \PYG{o}{=} \PYG{n}{current\PYGZus{}portfolio}\PYG{o}{.}\PYG{n}{set\PYGZus{}port\PYGZus{}variance}\PYG{p}{(}
\PYG{g+go}{  key = \PYGZsq{}percentchange\PYGZsq{}}
\PYG{g+go}{)}
\PYG{g+gp}{\PYGZgt{}\PYGZgt{}\PYGZgt{} }\PYG{n}{volatility} \PYG{o}{=} \PYG{n}{np}\PYG{o}{.}\PYG{n}{sqrt}\PYG{p}{(}\PYG{n}{variance}\PYG{p}{)}
\PYG{g+gp}{\PYGZgt{}\PYGZgt{}\PYGZgt{} }\PYG{n+nb}{print}\PYG{p}{(}\PYG{n}{volatility}\PYG{p}{)}
\PYG{g+go}{0.01345831069378136}
\PYG{g+gp}{\PYGZgt{}\PYGZgt{}\PYGZgt{} }\PYG{n}{mean} \PYG{o}{=} \PYG{n}{np}\PYG{o}{.}\PYG{n}{mean}\PYG{p}{(}\PYG{n}{current\PYGZus{}portfolio}\PYG{o}{.}\PYG{n}{market\PYGZus{}data}\PYG{p}{[}\PYG{l+s+s1}{\PYGZsq{}}\PYG{l+s+s1}{portfolio}\PYG{l+s+s1}{\PYGZsq{}}\PYG{p}{]}\PYG{p}{)}
\PYG{g+gp}{\PYGZgt{}\PYGZgt{}\PYGZgt{} }\PYG{n+nb}{print}\PYG{p}{(}\PYG{n}{mean}\PYG{p}{)}
\PYG{g+go}{0.0007375242310493472}
\end{sphinxVerbatim}

We calculated 1.3\% daily standard deviation or daily volatility, if the
distribution is normally distributed around zero, then we would expect
that 95\% of the data is contained within approximately 2 standard
deviations. We can visually confirm, as well as look to see if there are
other distributional aspects we can visually distinguish:

\fvset{hllines={, ,}}%
\begin{sphinxVerbatim}[commandchars=\\\{\}]
\PYG{g+gp}{\PYGZgt{}\PYGZgt{}\PYGZgt{} }\PYG{k+kn}{import} \PYG{n+nn}{matplotlib}\PYG{n+nn}{.}\PYG{n+nn}{pyplot} \PYG{k}{as} \PYG{n+nn}{plt}
\PYG{g+gp}{\PYGZgt{}\PYGZgt{}\PYGZgt{} }\PYG{n}{marketdata}\PYG{p}{[}\PYG{l+s+s1}{\PYGZsq{}}\PYG{l+s+s1}{portfolio}\PYG{l+s+s1}{\PYGZsq{}}\PYG{p}{]}\PYG{o}{.}\PYG{n}{plot}\PYG{o}{.}\PYG{n}{hist}\PYG{p}{(}\PYG{n}{bins}\PYG{o}{=}\PYG{l+m+mi}{20}\PYG{p}{,}\PYG{n}{title}\PYG{o}{=}\PYG{l+s+s1}{\PYGZsq{}}\PYG{l+s+s1}{Portfolio Historic Returns}\PYG{l+s+s1}{\PYGZsq{}}\PYG{p}{)}
\PYG{g+gp}{\PYGZgt{}\PYGZgt{}\PYGZgt{} }\PYG{n}{plt}\PYG{o}{.}\PYG{n}{axvline}\PYG{p}{(}\PYG{n}{temp} \PYG{o}{*} \PYG{l+m+mf}{1.96}\PYG{p}{,} \PYG{n}{color}\PYG{o}{=}\PYG{l+s+s1}{\PYGZsq{}}\PYG{l+s+s1}{r}\PYG{l+s+s1}{\PYGZsq{}}\PYG{p}{,} \PYG{n}{linestyle}\PYG{o}{=}\PYG{l+s+s1}{\PYGZsq{}}\PYG{l+s+s1}{\PYGZhy{}\PYGZhy{}}\PYG{l+s+s1}{\PYGZsq{}}\PYG{p}{)} \PYG{c+c1}{\PYGZsh{} if centered around zero, then}
\PYG{g+gp}{\PYGZgt{}\PYGZgt{}\PYGZgt{} }\PYG{n}{plt}\PYG{o}{.}\PYG{n}{axvline}\PYG{p}{(}\PYG{o}{\PYGZhy{}}\PYG{n}{temp} \PYG{o}{*} \PYG{l+m+mf}{1.96}\PYG{p}{,} \PYG{n}{color}\PYG{o}{=}\PYG{l+s+s1}{\PYGZsq{}}\PYG{l+s+s1}{r}\PYG{l+s+s1}{\PYGZsq{}}\PYG{p}{,} \PYG{n}{linestyle}\PYG{o}{=}\PYG{l+s+s1}{\PYGZsq{}}\PYG{l+s+s1}{\PYGZhy{}\PYGZhy{}}\PYG{l+s+s1}{\PYGZsq{}}\PYG{p}{)} \PYG{c+c1}{\PYGZsh{}}
\end{sphinxVerbatim}

\begin{figure}[htbp]
\centering
\capstart

\noindent\sphinxincludegraphics{{portfolio_returns}.png}
\caption{Portfolio Returns}\label{\detokenize{gettingstarted:id2}}\end{figure}

This distribution looks highly centered around zero, which could signal
kurtosis. This seems indicative of equity data, especially for daily
returns. Right now, a good place to start thinking about metric
parameterization is to assume normality and independence in daily
returns. While this assumption might not be very good or might vary
between security to security in the portfolio, which we can account for
in simulation or purely using historic returns to calculate risk
metrics, we can use this distribution assumption to quickly get a Value
at Risk metric over a time horizon.

The default time horizon is 10 days at a 95\% confidence level for the
\sphinxcode{\sphinxupquote{set\_port\_variance}} method, so if we look at the returned
\sphinxcode{\sphinxupquote{value\_at\_risk}}:

\fvset{hllines={, ,}}%
\begin{sphinxVerbatim}[commandchars=\\\{\}]
\PYG{g+gp}{\PYGZgt{}\PYGZgt{}\PYGZgt{} }\PYG{n+nb}{print}\PYG{p}{(}\PYG{n}{value\PYGZus{}at\PYGZus{}risk}\PYG{p}{)}
\PYG{g+go}{\PYGZhy{}0.083413941112170473}
\end{sphinxVerbatim}

This value is simply the standard deviation scaled by time, at the
critical value specified:
\begin{equation*}
\begin{split}VaR_{t, T} = \sigma * \sqrt{T-t} * Z^{*}_{p = \alpha}\end{split}
\end{equation*}
We can interpret this Value at Risk as being the lower bound of the 95\%
confidence interval for the 10 day distribution. For this portfolio, a
return over 10 days less than 8.3\% should occur 2.5\% of the time, on
average. To get the dollar value of the 10 Day Value at Risk, we would
just multiply this percent change by the current portfolio market value.

\fvset{hllines={, ,}}%
\begin{sphinxVerbatim}[commandchars=\\\{\}]
\PYG{g+gp}{\PYGZgt{}\PYGZgt{}\PYGZgt{} }\PYG{n}{dollar\PYGZus{}value\PYGZus{}at\PYGZus{}risk} \PYG{o}{=} \PYG{n}{value\PYGZus{}at\PYGZus{}risk} \PYG{o}{*} \PYG{p}{(}\PYG{n}{current\PYGZus{}portfolio}\PYG{o}{.}\PYG{n}{initial\PYGZus{}value} \PYG{o}{+} \PYG{n}{current\PYGZus{}portfolio}\PYG{o}{.}\PYG{n}{marked\PYGZus{}change}\PYG{p}{)}
\PYG{g+gp}{\PYGZgt{}\PYGZgt{}\PYGZgt{} }\PYG{n+nb}{print}\PYG{p}{(}\PYG{n}{dollar\PYGZus{}value\PYGZus{}at\PYGZus{}risk}\PYG{p}{)}
\PYG{g+go}{\PYGZhy{}991.20786223592188}
\end{sphinxVerbatim}

Similarly, we could interpret as over the a 10 day period, on average,
2.5\% of the time there could be an approximate loss over \$991.21 dollars
for this portfolio.


\subsubsection{Simulating the Portfolio}
\label{\detokenize{gettingstarted:simulating-the-portfolio}}

\chapter{Indices and tables}
\label{\detokenize{index:indices-and-tables}}\begin{itemize}
\item {} 
\DUrole{xref,std,std-ref}{genindex}

\item {} 
\DUrole{xref,std,std-ref}{modindex}

\item {} 
\DUrole{xref,std,std-ref}{search}

\end{itemize}



\renewcommand{\indexname}{Index}
\printindex
\end{document}